\documentclass{article}
\usepackage{arxiv}
\usepackage[utf8]{inputenc} % allow utf-8 input
\usepackage[T1]{fontenc}    % use 8-bit T1 fonts
\usepackage{hyperref}       % hyperlinks
\usepackage{url}            % simple URL typesetting
\usepackage{booktabs}       % professional-quality tables
\usepackage{amsfonts}       % blackboard math symbols
\usepackage{nicefrac}       % compact symbols for 1/2, etc.
\usepackage{microtype}      % micro typography
\usepackage{graphicx}
\usepackage{subcaption}
\usepackage[scaled]{beramono}
\usepackage[procnames]{listings}
\usepackage{color}
\usepackage{wrapfig}
\usepackage[toc,page]{appendix}
\usepackage{import}
\usepackage{verbatim}

\definecolor{keywords}{RGB}{255,0,90}
\definecolor{comments}{RGB}{0,0,113}
\definecolor{red}{RGB}{160,0,0}
\definecolor{green}{RGB}{0,150,0}
\DeclareUnicodeCharacter{3C9}{ }
\raggedbottom

\newcommand{\code}[2]{
    \hrulefill
    \subsection*{#1}
    \lstinputlisting[
        language=Python,
        basicstyle=\ttfamily\small,
        keywordstyle=\color{keywords},
        commentstyle=\color{comments},
        stringstyle=\color{red},
        breaklines=true,
        showstringspaces=false,
        numbers=left,
        stepnumber=1,
        procnamekeys={def,class}
    ]{#2}
    \vspace{2em}
}

\newcommand{\logoutput}[2]{
    \hrulefill
    \subsection*{#1}
    \lstinputlisting[
        language=Python,
        basicstyle=\ttfamily\small,
        commentstyle=\color{comments},
        stringstyle=\color{red},
        breaklines=true,
        showstringspaces=false,
        numbers=left,
        stepnumber=1,
        procnamekeys={Query variable,Evidence,Ordering}
    ]{#2}
    \vspace{2em}
}

\title{Human-Robot Interaction}
\renewcommand{\undertitle}{Robot learning}
\date{}

\author{
    J. Verhaert \\
    S1047220\\
    Master Artificial Intelligence | Intelligent Technology \\
    \texttt{joost.verhaert@student.ru.nl} \\
    \And
    T. Gelton \\
    S4480783\\
    Master Artificial Intelligence | Intelligent Technology \\
    \texttt{thijs.gelton@student.ru.nl} \\
}
\renewcommand{\headeright}{Robot learning}

\begin{document}
    \maketitle
    \pagebreak


    \section{Introduction}\label{sec:introduction}
    The goal of the second assignment of human-robot interaction is to incorporate robot learning techniques in HRI applications. The first part of this assignment targets to remind the students about the concepts and techniques of Neural Networks learned during the bachelor program of artificial intelligence. Last year, both of us followed this course in the pre-master. Therefore, the knowledge of this material was still good with us and understood these algorithms quickly. Part 2 of this assignment focuses on applying Mixed Density Networks (MDN) on the functions of the previous assignment regarding movements of the robot Noa. The concept of MDNs is a combination of deep neural networks and a mixture of distribution. The objective of the network is to learn the probability distribution of the output $y$ given the input $x: p(y|x)$ …...  I forgot what you said about important places of the camera (upper- left /& right)
    %- Describe the stages for completing the task


    \section{Methods and algorithms}\label{sec:methods-and-algorithms}
    %- Methods: Explain the methods/algorithms used and explain why


    \section{Results}\label{sec:results}
    %- Describe the results incorporating the advantages and limitations of the methods used.


    \section{Conclusion}\label{sec:conclusion}
    %- Draw your own conclusion about this robot learning approach

    % \cite{mirrazavi2018unified} <-- multi arm motion planning with MDN's. Usable as citation

    \bibliographystyle{ieeetr}
    \bibliography{main}
\end{document}
